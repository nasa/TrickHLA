\chapter{Upgrading Your Federate To Initiate a Federation Save}
\label{sec:hla_fed_save_setup}

Before a Trick federate can save itself, some upgrades to the Trick model
have to occur.

% ---------------------------------------------------------------------------
\section{Trick {\tt input} file update}
\label{sec:hla_fed_save_upg_input}

The {\tt input} file must specify a simulation initialization scheme 
of {\tt THLA\_MULTIPHASE\_INIT\_V2}, which represents the
{\em \href{file:IMSim_Multiphase_Init_Design_Document.pdf}
          {IMSim Multiphase Initialization Design with Late Joiners, Rejoiners and Federation Save \& Restore}}
\cite{trickhlaenv:IMSim-multiphase-init-design} scheme.

\begin{verbatim}
/* Use a simulation initialization scheme that supports save and restore. */
THLA.manager.sim_initialization_scheme = THLA_MULTIPHASE_INIT_V2;
\end{verbatim}

The user can specify where each checkpoint-ing and restore-ing Trick federate
is to store and reload the checkpoint files in {\tt TrickHLAFederate}'s
{\tt HLA\_save\_directory} variable in each input file. If the directory is not
specified in the {\tt input} file, it will be assigned by \TrickHLA\ to the
simulation's {\tt RUN} directory.

This is done because the checkpoint file name will be the same for all Trick
federates with no uniqueness identifiers added, i.e. something to distinguish
one file from another in the same directory, so it is recommended that the
{\tt HLA\_save\_directory} specify the Trick simulation's RUN directory.

If the user wishes to have the checkpoint files saved in another location, all
they have to do is specify the absolute path to the new location in 
the {\tt THLA.federate.HLA\_save\_directory} variable in the input file.

% ------------------------------------
\section{{\tt S\_define} file updates}

The updates needed in the {\tt S\_define} file necessary for {\tt TrickHLA}'s save and restore capability are provided in the
\TrickHLA\ {\tt sim\_object} in the {\tt THLA.sm} file, located in the
{\tt S\_modules} subdirectory. The specific lines of code are described in this section.

% ---------------------------------------------------------------------------
\subsection{Data Declarations}

All federates must be in {\tt freeze} mode at the same logical time in order to perform a federation save.
The same is true for a federation restore (although you can also perform a federation restore at startup).
This is accomplished by sending a special "freeze" interaction to all federates. A freeze interaction handler
is declared in the {\tt DATA STRUCTURE DECLARATIONS} section:

\begin{verbatim}
   TrickHLA: TrickHLAFreezeInteractionHandler freeze_ih;
\end{verbatim}

When performing a federation save via the {\tt start\_federation\_save()} call 
(see section~\ref{sec:prog_save_and_restore} {\tt Programmatic Save and Restore} below),
the time (optionally) and filename used for the save must be specified, either in a trick model or in the input file.
So a checkpoint\_time and checkpoint\_label variable are also declared:

\begin{verbatim}
   double checkpoint_time;
   char   checkpoint_label[256];
\end{verbatim}

% ------------------------------------
\subsection{Interactive Save and Restore}

The user may choose to freeze the federation by issuing a Trick freeze command, usually by clicking
the {\tt Freeze} button on the Trick Simulation Control Panel, or by an input file freeze command. The following
job will send a freeze interaction when a Trick freeze is commanded:

\begin{verbatim}
   /*
    * -- Coordinate federates going to freeze mode
    */
   P65534 (THLA_DATA_CYCLE_TIME, THLA_CHECK_PAUSE_JOB_OFFSET, logging) TrickHLA: THLA.federate.enter_freeze();
\end{verbatim}

Another job will assure that each federate freezes at the same logical time when the freeze interaction
is received:

\begin{verbatim}
   /*
    * -- Check to see if an interaction informed us that we are to
    *    FREEZE the sim before entering the next logical frame.
    */
   P65534 (THLA_DATA_CYCLE_TIME, logging) TrickHLA: THLA.federate.check_freeze_time();
\end{verbatim}

Once the federation is frozen by this means, the user can issue a Trick {\tt Dump Chkpnt} or {\tt Load Chkpnt} command
via the Trick Simulation Control Panel to initiate the federation save or restore, respectively.

The following two jobs get the federation running again when the Trick Simulation Control Panel {\tt Start} button is clicked:

\begin{verbatim}
   /*
    * -- Coordinate federates going to run mode
    */
   (freeze)   TrickHLA: THLA.federate.check_freeze();
   (unfreeze) TrickHLA: THLA.federate.exit_freeze();
\end{verbatim}

% ------------------------------------
\subsection{Programmatic Save and Restore}
\label{sec:prog_save_and_restore}

If the user wishes to trigger a federation save via a trick model or the input file, calling the {\tt start\_federation\_save()}
routine will send the freeze interaction and initiate the federation save when the freeze occurs 
(see section~\ref{sec:hla_trick_fed_save} {\tt Federation Save} below).

\begin{verbatim}
   /*
    * -- Initiate a federation save announcement if told to do so...
    */
   P1 (0.0, environment) TrickHLA: THLA.manager.start_federation_save(
      In const char * file_name = THLA.checkpoint_label );

   P1 (0.0, environment) TrickHLA: THLA.manager.start_federation_save_at_sim_time(
      In double freeze_sim_time = THLA.checkpoint_time,
      In const char * file_name = THLA.checkpoint_label );

   P1 (0.0, environment) TrickHLA: THLA.manager.start_federation_save_at_scenario_time(
      In double freeze_scenario_time = THLA.checkpoint_time,
      In const char * file_name = THLA.checkpoint_label );
\end{verbatim}

If a time value is not specified, or if the time specified has already past,
the current federation time will be used to determine when the next opportunity is to perform the
federation save.

The only {\tt TrickHLA} provided means of performing a restore programmatically is by setting the manager's {\tt restore\_federation} flag
(see section~\ref{sec:hla_trick_fed_restore} {\tt Federation Restore} below), in which case the restore occurs at startup.
A time cannot be specified for performing a restore.

% ------------------------------------
\subsection{Save and Restore Jobs}

When a federation save or restore has been initiated, the current {\tt TrickHLA} implementation depends
on asynchronous callbacks from the {\tt RTI} to guide the federation from one stage to another until the federation
save or restore is complete. {\tt TrickHLA} accomplishes this using a checkpoint class job and freeze class job (for save),
and a pre\_load\_checkpoint class job and freeze job (for restore).

\begin{verbatim}
   /*
    * -- Perform the federate save (checkpoint) or restore in FREEZE mode...
    */
   P1 (checkpoint)          TrickHLA: THLA.federate.setup_checkpoint();
   (freeze)                 TrickHLA: THLA.federate.perform_checkpoint();
   P1 (pre_load_checkpoint) TrickHLA: THLA.federate.setup_restore();
   (freeze)                 TrickHLA: THLA.federate.perform_restore();
\end{verbatim}



