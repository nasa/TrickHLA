%----------------------------------
\chapter{Preliminaries}\label{sec:preliminaries}
%----------------------------------

\section{Background}

\TrickHLA\ is a {\em glue layer} between Trick and HLA.
As such, using it to develop distributed simulations requires
that you have an understanding of Trick itself and to a lesser extent
be familiar with HLA terminology.
This section reviews the concepts that the \TrickHLA\ model is based on.

\section{Important Trick Concepts}

\paragraph{{\sdefine} Files.}
Trick is a simulation development environment that is used to build (compile, link, etc...)
executable images for your simulations.
The details of what your simulation consists of (which things are being simulated)
and what dynamic scenario is simulated (what happens during the simulation)
are specified by you in the so-called \sdefine file.

This file contains declarations of {\em sim objects} (the things)
and {\em jobs} (what happens).
The sim objects correspond to C/C++ data structures defined in models written by you,
and the jobs are C/C++ functions (or methods) which are also part of the models.

\TrickHLA\ is a Trick model and hence consists of sim objects and jobs that you assemble
into your \sdefine file alongside the objects and jobs that make up your particular
simulation.
Indeed, integrating \TrickHLA\ with your simulation consists mostly
(but not completely)
of pasting in an \sdefine snippet which provides much of the logic
(objects and jobs)
necessary for your simulation to participate in an HLA federation.

\paragraph{Input Files.}

Trick is based on a philosophy of {\em data driven} simulation.
Trick models are aggressively parameterized,
and the parameters are intended to be driven from initial data resident
in input files.
This permits significant variations of a particular simulation
to be run without actually rebuilding (recompiling and relinking)
the simulation itself.

A Trick input file is a text file consisting of name/value pairs of data:
the names specify specific model parameters,
and the values specify the initial data to be assigned to those paramters.
Trick has an input processor which parses the input file and makes sure that
the various variables in the simulation are initialized accordingly.

\TrickHLA\ is a Trick model, and so configuring it involves setting certain
\TrickHLA\ parameters in your sim input files.

\paragraph{Enabling/disabling jobs.}
In some of the simulations in this document,
we will want to enable/disable certain Trick jobs without
actually removing them from the \sdefine file,
since changing the \sdefine file requires that the simulation be recompiled.
This can be done from the input file by using the {\tt JOB} directive:

\begin{verbatim}
JOB job-name = [On|Off];
\end{verbatim}

where {\tt job-name} can be found in the list of all job-related
parameters in the Trick-generated file, {\tt S\_default.dat}.\footnote{
  This file is one of the files created by {\tt CP} when you build
  your simulation.
  It is located in the same directory as the \sdefine file.
}
Thus, to enable/disable a particular job, you just change the input value
in the directive between {\tt On} and {\tt Off}.
You need one such directive for each job you wish to disable.
Of course, all jobs without a corresponding {\tt JOB} directive are enabled.

\paragraph{Calling jobs at select times.}
In some of the simulations in this document,
rather than invoking Trick jobs periodically from the \sdefine file,
we want to invoke them at select times during the run.
This is handled by the {\tt READ} and {\tt CALL} directives:

\begin{verbatim}
READ = T;
CALL job-name;
\end{verbatim}

{\tt job-name} can be found in the list of all job-related
parameters in {\tt S\_default.dat}.
This syntax tells Trick to call the specified job at time {\tt T}.
If such a line occurs once in the input file (for only one value of {\tt T}),
then Trick will only invoke the specified job once at time, $t = {\tt T}$.

One subtle point must be made about the {\tt CALL} directive.
If your input file uses {\tt CALL} to invoke a job that is otherwise
{\em not} part of the Trick job schedule (as defined in the \sdefine file),
then you {\bf must} declare the job in the \sdefine with a frequency of zero.
This ensures that Trick generates the necessary code for the job
in spite of the fact that it is not in fact called by the Trick scheduler.
If you do not do this, you will get a runtime error during the execution
of your simulation.

\section{Important HLA Concepts}

\paragraph{Federations and Federates.}
An association of possibly distributed processes cooperating using HLA
is called a {\em federation}.
Processes may {\em join} a federation to participate,
and they may also {\em resign}.
The first federate to join a federation generally {\em creates} it,
and the last one to resign usually {\em destroys} it.

\paragraph{Objects and Interactions.}
There are two kinds of data in HLA: {\em objects} and {\em interactions.}

{\em Objects} are used to model data that are persistent over time,
entities which will evolve as the federation executes.
Objects are composed of {\em attributes,} and are as such very similar to
the classes of traditional object oriented languages.
Objects provide the abstract framework for how persistent data are structured.
Specific occurences of the data in a federation are refered to as
{\em instances}, and have federation-wide unique names.
It is these instances that persist during the execution of the federation.

{\em Interactions} are used to model events which contain data but are
not persistent; they are notifications.
Interactions are composed of {\em parameters}.\footnote{
  Parameters are to interactions as attributes are to objects.
}
Specific occurences of interactions are delivered to federates by the
HLA API as they are sent and/or received, and there is no concept of
an ``instance'' of an interaction.

\paragraph{Publish/subscribe.}
Federates that can generate values for particular instance attributes
are said to {\em publish} them.\footnote{
  Note that the use of the term publish in HLA only indicates a
  federate's {\em ability} to generate values and not the actual act of
  generating them.
}
Publishing is on a per-attribute basis.
Thus several federates may be involved in generating the values for
the attributes of a particular instance.
Furthermore, many federates may declare that they are publishers of a
particular attribute (i.e., that they have the ability to generate new
values for it); however, only one may actually generate new values:
this one federate is referred to as the {\em owner}.
When an owner generates new values for an attribute, it is said to
{\em update} them.
When a non-owner receives new values for an attribute from the owner,
the receiver is said to {\em reflect} the values.
During the execution of a federation,
ownership of a particular attribute may move from one federate to another.
This is called {\em ownership transfer}.
The act of giving up ownership of an attribute is called {\em divestiture}.
The act of assuming ownership of an attribute {\em acquisition}.

\paragraph{Send/receive.}
A federate may {\em send} a interaction to any other interested federates.
When the interaction arrives at the interested federates, they are said
to {\em receive} it.
These are on a per-interaction basis (not per-parameter).
When an interaction is sent, values for each of its parameters are
specified by the sender, and the entire interaction with each of its
parameters is delivered to any receivers.

\paragraph{Federation Object Model.}
HLA federations each have a specific {\em federation object model (FOM)}
that defines the structure of the types of objects and interactions
that may be exchanged by participating federates.
The information included in the FOM specifies the names and data types
of the shared object classes and attributes and interactions and parameters.
(It does not document specific object instances, however.)
The format of the FOM is not specified by the HLA standard, but the
Pitch implementation of HLA
(currently used for NASA/DSES HLA simulations)
uses an XML format.

\paragraph{Runtime Infrastructure.}
Most distributed computing envronments require some additional
processes in order to help coordinate communication between federates.
For HLA, this component is called the
{\em runtime infrastructure (RTI)}.
The Pitch RTI
(currently used for NASA/DSES HLA simulations)
is a single process which runs at a well-known TCP port number
on a well-known host on the network.
Each federate much be configured with this RTI host/port information
in order to join the federation.

\paragraph{Time Management.}
One of the unique capabilities HLA provides is a mechanism for keeping
all the federates in a federation synchronized.
In the HLA context, this means making sure that all federates see the
same sequence of data (instance attributes and interaction parameters)
at the same federation-time.
The HLA services behind this are collectively
referred to as {\em time management}.

The HLA concept of {\em lookahead} makes this possible:
each federate declares a lookahead (measured in units of time),
and any message sent by that federate (attribute update or interaction send)
must have a time stamp greater than or equal to the current time
plus the lookahead, i.e., the federate must be able to extrapolate the
value slightly into the future in order to update its value.
When data is delivered in this synchronized manner, it is said to be
{\em time stamp ordered}.
(If this synchronization is disabled, data delivery is said to be
{\em receive ordered}).

Federates participate in the HLA time management services by specifying
whether or not they are {\em time regulating} and {\em time constrained}.
Time regulating federates may generate TSO data, and the advance of
federation time proceeds only with the explicit agreement of such federates.
Time contrained federates may receive TSO data, but they do not have a voice
in whether or not time in the federation may advance.
Federates that are only time regulating are data sources.
Federates that are only time constrained are passive listeners to the data.
The most common case of for a federate to be both, in which case it may
generate and receive data.
HLA only increments federation time due to explicit requests from
federates -- {\em time advance requests}.
When the HLA runtime infrastructure acknowledges such requests, it is said
to provide {\em time advance grants}.

\section{The {\tt simplesine} Model}
\label{sec:simplesine-model}

In the following chapters,
we use a simple sine wave model, {\tt simplesine}, to illustrate \TrickHLA\ in action.
Understanding the model and how it is used in a Trick \sdefine file
is important for those chapters.
This section introduces \simplesine with that in mind.

% -----------------------------------------------------------------------
\subsection{Description}

The system modeled by \simplesine is an undamped harmonic oscillator,
the dynamics of which are governed by the differential equation
\begin{equation}
\ddot{x} + w^2 x = 0,
\label{eq:EOM}
\end{equation}
which has an analytic solution of the form
\begin{subequations}
\label{eq:analytic-equations}
\begin{align}
x(t)       &= A        \sin {(\omega t + \phi)}, \\
\dot{x}(t) &= A \omega \cos {(\omega t + \phi)}.
\end{align}
\end{subequations}

The relevant model data are
the dynamic {\em state}, $(x, \dot{x})$ and
the constant system {\em parameters}, $(A, \phi, \omega)$,
where the parameters are specified as inputs
and the state is calculated dynamically as simulation outputs.\footnote{
In this model, the initial state cannot be specified as inputs explicitly
but rather through the parameters $A$ and $\phi$.
}

The \simplesine model has functions that may be used to calculate the state
analytically based on equations~\ref{eq:analytic-equations}.
It can also propagate the state approximately
based on numerical integration of the differential equations.
The integration involves the calculation of the derivative of
the 2-vector $\boldsymbol{z}$ defined as

\begin{equation}
\label{eq:differential-equations}
  \boldsymbol{z}(t)
  \equiv \left\{
            \begin{array}{c}
              x(t)\\ \dot{x}(t)
            \end{array}
     \right\} \\
  = \left\{
            \begin{array}{r}
              A \sin{(\omega t + \phi)} \\ A \omega \cos{(\omega t + \phi)}
            \end{array}
     \right\}
\end{equation}

So that

\begin{equation}
\label{eq:derivative-equations}
  \dot{\boldsymbol{z}}(t)
  = \left\{
            \begin{array}{c}
              \dot{x}(t)\\ \ddot{x}(t)
            \end{array}
     \right\} \\
  = \left\{
            \begin{array}{r}
              \dot{x}(t)\\ - \omega^2 x(t)
            \end{array}
     \right\} \\
  = \left\{
            \begin{array}{r}
                    A w \cos{(\omega t + \phi)} \\
                    - A w^2 \omega \sin{(\omega t + \phi)}
            \end{array}
     \right\}
\end{equation}

% -------------------------------------
\subsection{Model}

The source code for the \simplesine model is organized into three directories:
{\tt data}, {\tt include} and {\tt src}.
The files in these directories are discussed below.

% ----------
\subsubsection{Include Files}

The \simplesine C/C++ include files are in directory {\tt simplesine/include}.
A list of the files is shown below.

{
\begin{center}
\scriptsize
\begin{tabular}{|l|l|}
\hline
{\em filename} \\
\hline
\hline
{\tt simplesine.h} \\
\hline
{\tt simplesine\_InteractionHandler.h} \\
\hline
{\tt simplesine\_LagCompensator.h} \\
\hline
{\tt simplesine\_Packing.h} \\
\hline
{\tt simplesine\_proto.h} \\
\hline
\end{tabular}
\end{center}
}

\paragraph{\tt simplesine.h}
This file declares the fundamental \simplesine data structures.
There is a single data structure that in turn holds state- and parameter-related
data structures.
In the simulations that follow, the {\tt simplesine\_T} structure will be
frequently declared as a {\tt sim\_object} in the \sdefine files
when the simulations need to model a sine wave.

The {\tt simplesine\_T} structure consists of a {\em state} substructure,
which holds $x$ and $\dot{x}$
and a parameters substructure which holds the constant {\em parameters,}
$A$, $\phi$ and $\omega$.
In this model, only the parameters
may be set from the input processor.
The state may only be calculated by a Trick job.
The purpose of this is to ensure that the state and parameters are never
initialized to inconsistent values.\footnote{
  Of course, this requires that developers remember to explicitly
  call {\tt simplesine\_calc()} as an initialization job for every
  \simplesine sim variable.
}

The complete file is shown in Appendix~\ref{sec:simplesine-h}.

\paragraph{\tt simplesine\_proto.h}
This file declares the C functions exported by the \simplesine model.
These functions may be used in an \sdefine file as Trick jobs.
Functions of particular interest are
\begin{itemize}
  \item{\tt simplesine\_calc()}, which calculates the state, $(x, \dot{x})$
    according to equations~\ref{eq:analytic-equations},\footnote{
      Since only the \simplesine parameters have default values,
      this job may be used as an initialization job to initialize
      the state from the parameters.
    }
  \item{\tt simplesine\_deriv()} and {\tt simplesine\_integ()}, which
    are used to numerically integrate equations~\ref{eq:differential-equations}
    using the standard Trick integration scheme,
  \item{\tt simplesine\_copyXXX()}, several routines which copy \simplesine
    data from one data structure to another, and
  \item{\tt simplesine\_calcError()}, which calculates the error between
one \simplesine state and the true values based on equations~\ref{eq:analytic-equations}.
\end{itemize}


The complete file is shown in Appendix~\ref{sec:simplesine-proto-h}.

\paragraph{\tt simplesine\_InteractionHandler.hh}
This file declares the \simplesine C++ class which acts as a
\TrickHLA\ {\em interaction handler}.
It is a subclass of {\tt TrickHLAInteractionHandler},
which is the \TrickHLA\ class which defines how interactions
are sent and received.

The complete file is shown in Appendix~\ref{sec:simplesine-InteractionHandler-hh}.

\paragraph{\tt simplesine\_LagCompensator.hh}
This file declares the \simplesine C++ class which acts as a
\TrickHLA\ {\em lag compensator}.
It is a subclass of {\tt TrickHLALagCompensator},
which is the \TrickHLA\ class which defines how federates may compensate
for HLA-time lags created as a result of sending data to remote federates
and transfering ownership between federates.

The complete file is shown in Appendix~\ref{sec:simplesine-LagCompensator-hh}.

\paragraph{\tt simplesine\_Packing.hh}
This file declares the \simplesine C++ class which may be optionally
used by developers to {\em pack} outbound data prior to sending via HLA
and {\em upack} is upon receipt from HLA.
It is a subclass of {\tt TrickHLAPacking},
which has {\tt pack()} and {\tt unpack()} virtual methods that implement
the application-specific packing and unpacking logic.

The complete file is shown in Appendix~\ref{sec:simplesine-Packing-hh}.


% ----------
\subsubsection{Source Files}

The \simplesine C/C++ source files are in the directory {\tt simplesine/src}.

The implementation of the \simplesine functions and classes is in
{\tt .c} and {\tt .cpp} files located in the {\tt simplesine/src} directory.
The C functions are mainly implemented one function per file\footnote{
  The copy functions
  ({\tt simplesine\_copyParams()},
  {\tt simplesine\_copyParams()}, and
  {\tt simplesine\_copyParams()})
  are located in a single file, {\tt simplesine\_copy.c}.
},
and the C++ classes are implemented one class per file.
The file names are shown in the table below:

{
\begin{center}
\scriptsize
\begin{tabular}{|l|l|}
\hline
{\em filename} & {\em implements what?} \\
\hline
\hline
{\tt simplesine\_calc.c} & {\tt simplesine\_copy()} \\
\hline
{\tt simplesine\_calcError.c} & {\tt simplesine\_calcError()} \\
\hline
{\tt simplesine\_compensate.c} & {\tt simplesine\_compensate()} \\
\hline
{\tt simplesine\_copy.c} & the {\tt simplesine\_copyXXX()} functions \\
\hline
{\tt simplesine\_deriv.c} & {\tt simplesine\_deriv()} \\
\hline
{\tt simplesine\_integ.c} & {\tt simplesine\_integ()} \\
\hline
{\tt simplesine\_propagate.c} & {\tt simplesine\_propagate()} \\
\hline
\hline
{\tt simplesine\_InteractionHandler.cpp} & the \TrickHLA\ interaction handler class \\
\hline
{\tt simplesine\_LagCompensator.cpp} & the \TrickHLA\ lag compensator class \\
\hline
{\tt simplesine\_Packing.cpp} & the \TrickHLA\ packing/unpacking class \\
\hline
\end{tabular}
\end{center}
}


% ----------
\subsubsection{Data Files}

This {\tt simplesine/data} directory
consists Trick input files for default \simplesine data.

Depending on how you build your \sdefine file, these files may be used
as ``fallback'' initializations for your data.
In the simulations that follow, the actual initial values
will often override the defaults in these files.\footnote{
  Sine the \simplesine state data are output-only,
  they cannot be set directly from the Trick input processor.
  Consequently, there is no default data file for
  {\tt simplesine\_state\_T}.
}

The files are shown below.

{
\begin{center}
\scriptsize
\begin{tabular}{|l|l|}
\hline
{\em filename} & {\em description} \\
\hline
\hline
{\tt integ.d}
  &
  Default numerical integration parameters.
  \\
\hline
{\tt simplesine.d}
  &
  Default \simplesine parameters with uninitialized state.
  \\
\hline
\hline
{\tt simplesine\_params.d}
  &
  Default \simplesine parameters.
  \\
\hline
\end{tabular}
\end{center}
}

\section{Federation Object Model}
\label{sec:FOM}

In the chapters that follow, various object instances and interactions are used.
Some of the simulations exchange data by putting \simplesine state and parameters
in a class instance.
Others exchange data by putting the parameters into an interaction.


This section shows the FOM snippets that define the relevant HLA class
and interaction.

\subsection{FOM structure}

The Pitch FOM is an XML file that has a structure shown in
Listing~\ref{list:FOM-structure}.
For each object class used by the simulation,
there must be a corresponding $\langle${\tt objectClass}$\rangle$ element
containing as many $\langle${\tt attribute}$\rangle$
subelements as it has attributes.
Similarly for each interaction class used by the simulation,
there must be a corresponding $\langle${\tt interactionClass}$\rangle$
element containing as many $\langle${\tt parameter}$\rangle$
subelements as it has parameters.\footnote{
  A full FOM includes other elements not shown in the listing
  and also declares object and interaction classes used by the underlying
  HLA infrastructure and not by the federates themselves.
  This document does not address how to build a FOM file from scratch.
}

\begin{lstlisting}[language=xml,caption={FOM structure},label={list:FOM-structure}]
<?xml version="1.0" encoding="UTF-8"?>
<!DOCTYPE objectModel SYSTEM "hla.dtd">
<objectModel ...>

  <!-- Declaration of all object classes known to the federation -->
  <objects>
    <objectClass name="..." ...>
      <attribute name="..." ... />
      ...
    </objectClass>
    ...
  </objects>

  <!-- Declaration of all interactions classes known to the federation -->
  <interactions>
    <interactionClass name="..." ...>
      <parameter dataType="..." name="..." />
      ...
    </interactionClass>
  </interactions>

  ...
\end{lstlisting}

\subsection{Object class declaration}

The XML definition of a class consists of a single XML element,
$\langle$~{\tt objectClass}~$\rangle$
containing one child element
$\langle$~{\tt attribute}~$\rangle$
for each attribute.

The class used in some of the following simulations is named
{\tt SimplesineStateAndParameters} and is defined by the following XML snippet.
Lines 2-7 declare one attribute for each of
the current time, the state and parameter values
($t$, $x$, $\dot{x}$, $A$, $\phi$ and $\omega$).

\begin{lstlisting}[caption={FOM snippet defining a class},label={list:FOM-snippet-class}]
<objectClass name="SimplesineStateAndParameters" sharing="Neither">
  <attribute dimensions="NA" name="Time" order="TimeStamp" transportation="HLAreliable"/>
  <attribute dimensions="NA" name="Value" order="TimeStamp" transportation="HLAreliable"/>
  <attribute dimensions="NA" name="dvdt" order="TimeStamp" transportation="HLAreliable"/>
  <attribute dimensions="NA" name="Phase" order="TimeStamp" transportation="HLAreliable"/>
  <attribute dimensions="NA" name="Frequency" order="TimeStamp" transportation="HLAreliable"/>
  <attribute dimensions="NA" name="Amplitude" order="TimeStamp" transportation="HLAreliable"/>
</objectClass>
\end{lstlisting}

\subsection{Interaction class declaration}

Similarly, the interactions used in some of the following simulations
is named {\tt SimplesineParameters} and is defined in the following XML
snippet.
Lines 3-5 declare one parameter for each of the state and parameter values
($A$, $\phi$ and $\omega$).

\begin{lstlisting}[caption={FOM snippet defining a interaction},label={list:FOM-snippet-interaction}]
<interactionClass dimensions="NA" name="SimplesineParameters"
                  order="TimeStamp" sharing="Neither" transportation="HLAreliable">
  <parameter dataType="HLAfloat64LE" name="A"/>
  <parameter dataType="HLAfloat64LE" name="w"/>
  <parameter dataType="HLAfloat64LE" name="phi"/>
</interactionClass>
\end{lstlisting}


\subsection{Simulation Configuration declaration}

The following XML snippet defines the HLA class used to capture
\TrickHLA\ simulation configuration information.
\TrickHLA\ requires that a simulation configuration class be defined in
the FOM, and the snippet below shows the class definition used by
DSES simulations.
(This document does not address how to design simulations which use a
different simulation configuration class.)

\begin{lstlisting}[caption={FOM snippet defining a interaction},label={list:FOM-snippet-config}]
<objectClass name="SimulationConfiguration" sharing="PublishSubscribe">
  <attribute dataType="HLAlogicalTime" dimensions="NA" name="run_duration" order="Receive"
             semantics="Duration of run" sharing="PublishSubscribe"
             transportation="HLAreliable" updateType="Static"/>
  <attribute dataType="HLAinteger32LE" dimensions="NA" name="number_of_federates" order="Receive"
             semantics="Number of required federates for run" sharing="PublishSubscribe"
             transportation="HLAreliable" updateType="Static"/>
  <attribute dataType="HLAinteger32LE" dimensions="NA" name="start_year" order="Receive"
             semantics="Year at start of run" sharing="PublishSubscribe"
             transportation="HLAreliable" updateType="Static"/>
  <attribute dataType="HLAinteger32LE" dimensions="NA" name="start_seconds" order="Receive"
             semantics="Starting time of run in seconds-of-year" sharing="PublishSubscribe"
             transportation="HLAreliable" updateType="Static"/>
  <attribute dataType="HLAunicodeString" dimensions="NA" name="owner" order="Receive"
             semantics="Federation publishing object" sharing="PublishSubscribe"
             transportation="HLAreliable"/>
  <attribute dataType="HLAunicodeString" dimensions="NA" name="scenario" order="Receive"
             semantics="Scenario being simulated." sharing="PublishSubscribe"
             transportation="HLAreliable"/>
  <attribute dataType="HLAunicodeString" dimensions="NA" name="mode" order="Receive"
             semantics="Mode of simulation run." sharing="PublishSubscribe"
             transportation="HLAreliable"/>
  <attribute dataType="HLAunicodeString" dimensions="NA" name="required_federates" order="Receive"
             semantics="Comma-separated list of required federates." sharing="PublishSubscribe"
             transportation="HLAreliable"/>
</objectClass>
\end{lstlisting}


