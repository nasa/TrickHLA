\chapter{Federation Save}
\label{sec:hla_trick_fed_save}

A federation save must occur in {\tt freeze} mode, initiated via a freeze interaction (see
{\em \href{file:IMSim_Multiphase_Init_Design_Document.pdf}
          {IMSim Multiphase Initialization Design with Late Joiners, Rejoiners and Federation Save \& Restore}}
\cite{trickhlaenv:IMSim-multiphase-init-design}
chapter 14, for details.)
Once a federate receives this interaction, or reaches the
interaction frame if this is the federate that sent the freeze interaction,
it will go into {\tt freeze} mode at the bottom of the next execution frame.

If this is the federate that sent the interaction, {\tt TrickHLA} registers a {\tt FEDSAVE\_v2}
synchronization point which all federates must acknowledge, and the federation must
synchronize on, before proceeding with the federation save process. This is to avoid a
race condition in which the federate that sent the interaction would request the
federation save and any federate not in the same state (i.e. still executing the
frame and not in {\tt freeze} mode) will emit an exception when going into the time
advance state (via calling {\tt wait\_for\_time\_advance\_grant()} routine) because
the {\tt RTI} is already in federation save mode. This would be a fatal error since
the non-frozen federate's execution would timeout waiting for a time
advance grant.

Each federate, once in {\tt freeze} mode, shall achieve the {\tt FEDSAVE\_v2}
synchronization point with the {\tt RTI} and wait until signaled by the {\tt RTI} to
begin its save.

% ------------------------------------
\section{Interactive Save }
\label{sec:interactive_save}

Perhaps the most straightforward way to perform a federation save is via the Trick Simulation Control Panel.
Simply click the {\tt Freeze} button on a federate's simulation control panel, and a freeze interaction is sent
so that all federates will freeze at the same time (usually one or two lookahead\_time frames after the freeze
click). Then click the {\tt Dump Chkpnt} button to trigger the federation save. A window will pop up where you 
can enter the user file name for the checkpoint file to be dumped. Each federate will dump a file of the form:

\begin{verbatim}
<federation_name>_<user_file_name>
\end{verbatim}

in the directory specified by {\tt THLA.federate.HLA\_save\_directory}, which defaults to the {\tt RUN}
directory. Note that {\tt TrickHLA} automatically prepends the federation name to the given user file name.
There is also another file created named {\tt <federation\_name>\_<user\_file\_name>.running\_feds} which
is for {\tt TrickHLA}'s internal use. All federates will dump these two files in their respective directories.

The {\tt RTI} itself will also save its own relevant data in a separate directory, which should be transparent to the user, 
but is configurable (see the {\tt RTI} documentation for more information).

Simply click the {\tt Start} button on the simulation control panel when you are ready to continue execution.

IMPORTANT: Use the same federate's Trick simulation control panel when clicking {\tt Freeze} and {\tt Start}.
Each federate may have its own control panel, but you must use the same control panel that you clicked {\tt Freeze}
on to then click {\tt Start} on. If you use a different control panel to click the {\tt Start} button, the simulation will most
likely not be able to continue.

% ------------------------------------
\section{Programmatic Save }
\label{sec:prog_save}

The {\tt TrickHLAManager::start\_federation\_save()}
routine can be used to initiate the federation
save from any Trick federate. There are three flavors of this routine.
The first version of the routine will send a freeze interaction to pause the
simulation at the bottom of the next frame, which is the earliest time a
coordinated federation save can occur:

\begin{verbatim}
void start_federation_save( const char * file_name ) ;
\end{verbatim}

The second version of the routine will send a freeze interaction to pause the
simulation at the bottom of the frame at the user-supplied simulation time,
which could be a time later than the next frame:

\begin{verbatim}
void start_federation_save_at_sim_time( double freeze_sim_time,
                                        const char * file_name ) ;
\end{verbatim}

The third version of the routine will send a freeze interaction to pause the
simulation at the bottom of the frame at the user-supplied scenario time,
which could be a time later than the next frame:

\begin{verbatim}
void start_federation_save_at_scenario_time( double freeze_scenario_time,
                                             const char * file_name ) ;
\end{verbatim}

The following are examples of triggering the federation
save at simulation time 8.0 via the Trick input file. Note that the freeze (and therefore
the save) will occur a couple of lookahead\_time frames after 8.0:

\begin{verbatim}
read=8.0
THLA.checkpoint_label = "checkpoint.8.000";
CALL THLA.THLA.manager.start_federation_save( THLA.checkpoint_label );

   -- or --

THLA.checkpoint_label = "checkpoint.8.000";
THLA.checkpoint_time = 8.0;
CALL THLA.THLA.manager.start_federation_save_at_sim_time( 
                                  THLA.checkpoint_time, THLA.checkpoint_label );

   -- or --

BEGIN_EVENT checkpoint_1 {
  CONDITION: sys.exec.out.time == 8.0 ;
ACTION {
   THLA.checkpoint_label = "checkpoint.8.000";
   CALL THLA.THLA.manager.start_federation_save( THLA.checkpoint_label );
   }
};
\end{verbatim}

The location and name of the files created are the same as described in the above section~\ref{sec:interactive_save}
 {\tt Interactive Save}.

NOTE: You do not have to specify the federation name prepended to the checkpoint file name.
{\tt TrickHLA} will automatically prepend the federation name to the file name you supply if need be.
So in the above example, if the federation 
name is {\tt MyFederation}, then the file that will be saved is {\tt MyFederation\_checkpoint.8.000} in the {\tt RUN} directory.

By default, after the save has taken place each federate simulation will remain in {\tt freeze} mode until commanded
by the user to go to run (e.g. via the Trick simulation control panel). If this is not the desired behavior, then set the federate {\tt unfreeze\_after\_save} flag in 
each federate's Trick input file:

\begin{verbatim}
THLA.federate.unfreeze_after_save = 1 ;
\end{verbatim}

Setting this flag for all federates will cause the federation to resume execution immediately after the save has completed.

