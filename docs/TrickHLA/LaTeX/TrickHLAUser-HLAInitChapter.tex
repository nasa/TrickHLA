\chapter{Initialization}
\label{sec:hla-init}

As was discussed in Section~\ref{sec:hla-join},
there is more to initializing an HLA federate using
\TrickHLA\ than just joining the federation.
In addition to joining,
\TrickHLA\ {\bf requires} that you follow the
{\em multiphase initialization process}.
This chapter illustrates how to do that.

% -----------------------------------------------------------------------
\section{What is multiphase initialization?}

The initialization of the components in a distributed simulation can be
more complex than simply letting each components initialize itself.
Initialization dependencies between components might require that the
components {\em partially} initialize themselves.
The mechanism provided by \TrickHLA\ for this called {\em multiphase
initialization}.
It works as follows.

The developers of a distributed simulation must agree beforehand
on the number of {\em initialization phases} to be executed by each
simulation during startup.
Furthermore, each phase has a unique name.
\TrickHLA\ then provides a means for each simulation developer to
schedule jobs during each phase of the initialization.
In particular, data calculated during one phase may be shared with
other federates for use during their subsequent initialization phases.

% -----------------------------------------------------------------------
\section{\tt DSESSimConfig}

The \TrickHLA\ mechanism for doing this is part of the {\tt THLA\_INIT}
sim object that was introduced in Chapter~\ref{sec:hla-join}.
In that and all the subsequent examples so far in this guide,
the {\tt THLA\_INIT} object consisted only of a single
{\tt DSESSimConfig} variable.
This variable is a fundamental part of the \TrickHLA\ infrastructure,
and indeed, \TrickHLA\ will not function without the variable.

However, in order to participate in the multiphase initialization process,
each simulation must also define the logic to be executed during each
phase of the initialization.
This is done by
\begin{itemize}
\item{
  Setting the {\tt THLA\_INIT} variables according to how many
  phases there are, and
}
\item{
  adding phase-specific jobs to the {\tt THLA\_INIT} sim object.
}
\end{itemize}

% -----------------------------------------
\subsection{{\tt THLA\_INIT} inputs for multiphase initialization}

The \TrickHLA\ implemenation of multiphase initialization is based on
HLA {\em synchronization points}.
Federates may use synchronization points as {\em barriers} where all must
gather before any are allowed to continue.
\TrickHLA\ uses them in this fashion to make sure that all participating
simulations have completed phase-$i$ processing before any of the
simulations proceed to phase-$i+1$.

Upon agreement among all the participants in the federation as to how
many initialization phases are necessary and what each is named,
the input file needs only include an entry for the federate object
which specifies the name of each phase.
For example, if a simulation requires two phases, named {\em Phase1}
and {\em Phase2}, then the input file would include a line of the form

\begin{verbatim}
THLA.federate.multiphase_init_sync_points = "Phase1, Phase2";
\end{verbatim}

Similarly, for three phases named {\em A}, {\em B} and {\em C},
the input file would include

\begin{verbatim}
THLA.federate.multiphase_init_sync_points = "A, B, C";
\end{verbatim}


% -----------------------------------------
\subsection{{\tt THLA\_INIT} jobs for multiphase initialization}

The tasks to be carried out in each phase of initialization are
\begin{itemize}
\item{
  Execute some application-specific initialization code.
}
\item{
  Share the results of that initialization code between the federates.
  (This involves sending locally-calculated intialization data out to
  other federates and receviing remotely-calculated initialization data
  from other federates.)
}
\item{
  Execute some application-specific post-initialization code.
}
\end{itemize}

Accordingly, each phase consists of Trick jobs which make calls to
the following functions.

\begin{itemize}
\item{{\tt THLA.manager.send\_init\_data()}}
\item{{\tt THLA.manager.receive\_init\_data()}}
\item{{\tt THLA.manager.wait\_for\_init\_sync\_point()}}
\end{itemize}

The arguments to the {\tt send\_init\_data()} and {\tt receive\_init\_data()}
methods are the names of Trick variables defined in the \sdefine file.
If there are many data to send, there will be correspondingly many
jobs invoking {\tt send\_init\_data()}
and similarly for receiving data.
The argument to the {\tt wait\_for\_init\_sync\_point()} method
is the name of the corresponding phase as defined in the sim config object.
Invocation of the {\tt wait\_for\_init\_sync\_point()} method
tells \TrickHLA\ that the simulation has completed the corresponding
initialization phase.
\TrickHLA\ ensures that no simulations will proceed beyond this point
until all the participating simulations have reached it.\footnote{
  This is implemented with HLA {\em synchronization points},
  whence the name of the method.
}

This is illustrated below for two-phase initialization.
Multiphase initialization is similar, except that there are more
send/receive/wait-for sequences,
one for each phase defined in the sim config object.

\begin{lstlisting}[caption={Multiphase initialization jobs}]
   //--------------------------------------------------------------------------
   // NOTE: Initialization phase numbers must be greater than P10 so that the 
   // initialization jobs run after the P10 THLA.manager.initialize() job.
   //--------------------------------------------------------------------------

   //
   // PHASE 1
   //
   P100 (initialization) TrickHLA:
      THLA.manager.send_init_data( In const char * obj_instance_name = "...name-of-trick-variable..." );
   P100 (initialization) TrickHLA:
      THLA.manager.receive_init_data( In const char * obj_instance_name = "...name-of-trick-variable..." );
   // ...Add optional application-specific post-phase-1 processing here.
   P100 (initialization) TrickHLA:
      THLA.manager.wait_for_init_sync_point( In const char * syc_point_label   = "...name-of-phase-1..." );

   //
   // PHASE 2
   //
   P200 (initialization) TrickHLA:
      THLA.manager.send_init_data( In const char * obj_instance_name = "...name-of-trick-variable..." );
   P200 (initialization) TrickHLA:
      THLA.manager.receive_init_data( In const char * obj_instance_name = "...name-of-trick-variable..." );
   // ...Add optional application-specific post-phase-1 processing here.
   P200 (initialization) TrickHLA:
      THLA.manager.wait_for_init_sync_point( In const char * syc_point_label   = "...name-of-phase-2..." );
\end{lstlisting}

% -----------------------------------------
\subsection{{\tt THLA\_INIT} jobs for one-phase initialization}

If true multiphase initialization is overkill for your simulation,
it is possible to define {\tt THLA\_INIT} jobs that just carry out
traditional single-step initalization in a slightly simpler way
than going through the steps described above.

The tasks to be carried out in single-step initialization are
\begin{itemize}
\item{
  Execute some application-specific initialization code.
}
\item{
  Share the results of that initialization code between the federates.
}
\item{
  Execute some application-specific post-initialization code.
}
\end{itemize}

To do this, the following jobs can be inserted into the {\tt THLA\_INIT}
sim object.

\begin{lstlisting}[caption={One-phase initialization jobs}]

   // Application-specific initialization code should have been
   // executed already in some previous Trick initialization job

   // Tell TrickHLA to "share" initialization data.
   P100 (initialization) TrickHLA: THLA.manager.send_init_data();
   P100 (initialization) TrickHLA: THLA.manager.receive_init_data();

   // ...Add application-specific post-initialization code here if necessary.

   // Tell TrickHLA that we're finished with the initialization process.
   P100 (initialization) TrickHLA: THLA.manager.clear_init_sync_points();
\end{lstlisting}



