\chapter{Joining a Federation}
\label{sec:hla-join}

In this chapter, we present a Trick simulation which uses
\TrickHLA\ to join an HLA federation.\footnote{
  The \TrickHLA\ initialization process is more complex than this
  simulation illustrates.
  The only focus here is on using \TrickHLA\ to enable HLA in a Trick
  simulation.
  The full \TrickHLA\ {\em multi-phase initialzation process} is discussed
  in Section~\ref{sec:hla-init} on page~\pageref{sec:hla-init}.
}
If the specified federation exists,
the simulation joins it.
If the federation does not exist, \TrickHLA\ takes care of creating it.

The simulation
is virtually identical to {\tt SIM\_simplesine\_pubsub} presented in
Section~\ref{sec:simplesine-model}, except that this simulation
joins (or creates) an HLA federation.
In particular, there is no distributed publish/subscribe;
the publisher and subscriber are both resident in this single simulation
process, and they exchange data to each other locally.

The main tasks to HLA-enable an existing Trick simulation are to
insert some ``standard'' \TrickHLA\ {\tt sim\_object}s
into the \sdefine file, and
configure the input file with new \TrickHLA\ parameters.
The following sections illustrate what is involved.

\section{\tt SIM\_simplesine\_hla\_join}
\label{sec:SIM-simplesine-hla-join-sdefine}

The first step in enabling \TrickHLA,
is inserting two new {\tt sim\_object}s into the \sdefine file:
one which contains the main \TrickHLA\ execution
framework (a bunch of jobs that automate the \TrickHLA\ process)
and
one which contains object and jobs related to simulation configuration
and initialization.
The first of these, {\tt THLA}, is shown below
and can be pasted in verbatim.\footnote{
  Some comments have been removed from this object to reduce page space.
}
In most cases, very few modifications to this {\tt sim\_object} should
be necessary.

Examination of the code reveals that \TrickHLA\ is composed of three objects:
the {\em mangager}, the {\em federate}, and the {\em federate ambassador}.
The \TrickHLA\ infrastructure is the base manager and federate jobs
that follow.
In most cases, it is possible to view these objects and their
jobs as black boxes: just paste the object into the \sdefine file.

\begin{lstlisting}[caption={The {\tt THLA sim\_object}},label={list:THLA-sim-object}]
sim_object {
   TrickHLA: TrickHLAFreezeInteractionHandler freeze_ih;
   TrickHLA: TrickHLAFedAmb   federate_amb;
   TrickHLA: TrickHLAFederate federate;
   TrickHLA: TrickHLAManager  manager;
   double checkpoint_time;
   char   checkpoint_label[256];

   // Initialization jobs
   P1 (initialization) TrickHLA: THLA.manager.print_version();
   P1 (initialization) TrickHLA: THLA.federate.fix_FPU_control_word();
   P60 (initialization) TrickHLA: THLA.federate_amb.initialize(
      In TrickHLAFederate * federate = &THLA.federate,
      In TrickHLAManager  * manager  = &THLA.manager );
   P60 (initialization) TrickHLA: THLA.federate.initialize(
      Inout TrickHLAFedAmb * federate_amb = &THLA.federate_amb );
   P60 (initialization) TrickHLA: THLA.manager.initialize(
      In TrickHLAFederate * federate = &THLA.federate );
   P65534 (initialization) TrickHLA: THLA.manager.initialization_complete();
   P65534 (initialization) TrickHLA: THLA.federate.check_pause_at_init( 
      In const double check_pause_delta = THLA_CHECK_PAUSE_DELTA );

   // Checkpoint related jobs
   P1 (checkpoint)          TrickHLA: THLA.federate.setup_checkpoint();
   (freeze)                 TrickHLA: THLA.federate.perform_checkpoint();
   P1 (pre_load_checkpoint) TrickHLA: THLA.federate.setup_restore();
   (freeze)                 TrickHLA: THLA.federate.perform_restore();

   // Freeze jobs
   (freeze)   TrickHLA: THLA.federate.check_freeze();
   (unfreeze) TrickHLA: THLA.federate.exit_freeze();

   // Scheduled jobs
   P1 (THLA_DATA_CYCLE_TIME, environment) TrickHLA: THLA.federate.wait_for_time_advance_grant();
   P1 (THLA_INTERACTION_CYCLE_TIME, environment) TrickHLA: THLA.manager.process_interactions();
   P1 (THLA_DATA_CYCLE_TIME, environment) TrickHLA: THLA.manager.process_deleted_objects();
   P1 (0.0, environment) TrickHLA: THLA.manager.start_federation_save(
      In const char * file_name = THLA.checkpoint_label );
   P1 (0.0, environment) TrickHLA: THLA.manager.start_federation_save_at_sim_time(
      In double freeze_sim_time = THLA.checkpoint_time,
      In const char * file_name = THLA.checkpoint_label );
   P1 (0.0, environment) TrickHLA: THLA.manager.start_federation_save_at_scenario_time(
      In double freeze_scenario_time = THLA.checkpoint_time,
      In const char * file_name = THLA.checkpoint_label );
   P1 (THLA_DATA_CYCLE_TIME, environment) TrickHLA: THLA.manager.receive_cyclic_data(
      In double current_time = sys.exec.out.time );
   P65534 (THLA_DATA_CYCLE_TIME, logging) TrickHLA: THLA.manager.send_cyclic_data(
      In double current_time = sys.exec.out.time );
   P65534 (THLA_DATA_CYCLE_TIME, logging) TrickHLA: THLA.manager.send_requested_data(
      In double current_time = sys.exec.out.time );
   P65534 (THLA_DATA_CYCLE_TIME, logging) TrickHLA: THLA.manager.process_ownership();
   P65534 (THLA_DATA_CYCLE_TIME, logging) TrickHLA: THLA.federate.time_advance_request();
   P65534 (THLA_DATA_CYCLE_TIME, logging) TrickHLA: THLA.federate.check_freeze_time();
   P65534 (THLA_DATA_CYCLE_TIME, THLA_CHECK_PAUSE_JOB_OFFSET, logging) TrickHLA: THLA.federate.check_pause( 
      In const double check_pause_delta = THLA_CHECK_PAUSE_DELTA );
   P65534 (THLA_DATA_CYCLE_TIME, THLA_CHECK_PAUSE_JOB_OFFSET, logging) TrickHLA: THLA.federate.enter_freeze();

   // Shutdown jobs
   P65534 (shutdown) TrickHLA: THLA.manager.shutdown();
} THLA;
\end{lstlisting}

In addition, the following {\tt THLA\_INIT} object should be put
into the \sdefine file.
The version shown here is sufficient for now.\footnote{
  We will discuss this object in more detail when we address
  \TrickHLA\ multi-phase initialization.
}

\begin{lstlisting}[caption={The {\tt THLA\_INIT sim\_object}},label={list:THLA-INIT-sim-object}]
sim_object {
   // The DSES simulation configuration.
   simconfig: DSESSimConfig  dses_config;

   // Generally, initialization jobs will go here, but not for this example.
   
   // Clear remaining initialization sync-points.
   P100 (initialization) TrickHLA: THLA.manager.clear_init_sync_points();
} THLA_INIT;
\end{lstlisting}

\section{Input Files}

The second step in enabling \TrickHLA\ is to set its parameters in
a Trick input file.
The input file for {\tt SIM\_simplesine\_hla\_join} is shown below.

\begin{lstlisting}[caption={{\tt SIM\_simplesine\_hla\_join} input file},label={list:SIM-simplesine-hla-join-input}]
#include "S_properties"
#include "S_default.dat"
#include "Log_data/states.d"
#include "Modified_data/realtime.d"
#include "Modified_data/publisher.d"
#include "Modified_data/subscriber.d"

stop =32.5;

//
// Basic RTI/federation connection info
//

// Configure the CRC for the Pitch RTI.
THLA.federate.local_settings = "crcHost = localhost\n crcPort = 8989";

THLA.federate.name            = "pubsub_join";
THLA.federate.FOM_modules     = "FOM.xml";
THLA.federate.federation_name = "simplesine";

THLA.federate.lookahead_time   = THLA_DATA_CYCLE_TIME;
THLA.federate.time_regulating  = true;
THLA.federate.time_constrained = true;
THLA.federate.multiphase_init_sync_points = "Phase1, Phase2";

THLA.federate.enable_known_feds      = true;
THLA.federate.known_feds_count       = 1;
THLA.federate.known_feds             = alloc(THLA.federate.known_feds_count);
THLA.federate.known_feds[0].name     = "pubsub_join";
THLA.federate.known_feds[0].required = true;

// TrickHLA debug messages.
THLA.manager.debug_handler.debug_level = THLA_LEVEL2_TRACE;

//
// SimConfig initialization
//

// DSES simulation configuration.
THLA_INIT.dses_config.owner              = "pubsub_join";
THLA_INIT.dses_config.run_duration       = 15.0;
THLA_INIT.dses_config.num_federates      = 1;
THLA_INIT.dses_config.required_federates = "pubsub_join";
THLA_INIT.dses_config.start_year         = 2007;
THLA_INIT.dses_config.start_seconds      = 0;
THLA_INIT.dses_config.scenario           = "Nominal";
THLA_INIT.dses_config.mode               = "Unknown";

// Simulation Configuration for DSES Multi-phase Initialization.
THLA.manager.sim_config.FOM_name         = "SimulationConfiguration";
THLA.manager.sim_config.name             = "SimConfig";
THLA.manager.sim_config.packing          = &THLA_INIT.dses_config;
THLA.manager.sim_config.attr_count       = 8;
THLA.manager.sim_config.attributes       = alloc(THLA.manager.sim_config.attr_count);

THLA.manager.sim_config.attributes[0].FOM_name     = "owner";
THLA.manager.sim_config.attributes[0].trick_name   = "THLA_INIT.dses_config.owner";
THLA.manager.sim_config.attributes[0].publish      = true;
THLA.manager.sim_config.attributes[0].subscribe    = true;
THLA.manager.sim_config.attributes[0].rti_encoding = THLA_UNICODE_STRING;

THLA.manager.sim_config.attributes[1].FOM_name     = "run_duration";
THLA.manager.sim_config.attributes[1].trick_name   = "THLA_INIT.dses_config.run_duration_microsec";
THLA.manager.sim_config.attributes[1].publish      = true;
THLA.manager.sim_config.attributes[1].subscribe    = true;
THLA.manager.sim_config.attributes[1].rti_encoding = THLA_LITTLE_ENDIAN;

THLA.manager.sim_config.attributes[2].FOM_name     = "number_of_federates";
THLA.manager.sim_config.attributes[2].trick_name   = "THLA_INIT.dses_config.num_federates";
THLA.manager.sim_config.attributes[2].publish      = true;
THLA.manager.sim_config.attributes[2].subscribe    = true;
THLA.manager.sim_config.attributes[2].rti_encoding = THLA_LITTLE_ENDIAN;

THLA.manager.sim_config.attributes[3].FOM_name     = "required_federates";
THLA.manager.sim_config.attributes[3].trick_name   = "THLA_INIT.dses_config.required_federates";
THLA.manager.sim_config.attributes[3].publish      = true;
THLA.manager.sim_config.attributes[3].subscribe    = true;
THLA.manager.sim_config.attributes[3].rti_encoding = THLA_UNICODE_STRING;

THLA.manager.sim_config.attributes[4].FOM_name     = "start_year";
THLA.manager.sim_config.attributes[4].trick_name   = "THLA_INIT.dses_config.start_year";
THLA.manager.sim_config.attributes[4].publish      = true;
THLA.manager.sim_config.attributes[4].subscribe    = true;
THLA.manager.sim_config.attributes[4].rti_encoding = THLA_LITTLE_ENDIAN;

THLA.manager.sim_config.attributes[5].FOM_name     = "start_seconds";
THLA.manager.sim_config.attributes[5].trick_name   = "THLA_INIT.dses_config.start_seconds";
THLA.manager.sim_config.attributes[5].publish      = true;
THLA.manager.sim_config.attributes[5].subscribe    = true;
THLA.manager.sim_config.attributes[5].rti_encoding = THLA_LITTLE_ENDIAN;

THLA.manager.sim_config.attributes[6].FOM_name     = "scenario";
THLA.manager.sim_config.attributes[6].trick_name   = "THLA_INIT.dses_config.scenario";
THLA.manager.sim_config.attributes[6].publish      = true;
THLA.manager.sim_config.attributes[6].subscribe    = true;
THLA.manager.sim_config.attributes[6].rti_encoding = THLA_UNICODE_STRING;

THLA.manager.sim_config.attributes[7].FOM_name     = "mode";
THLA.manager.sim_config.attributes[7].trick_name   = "THLA_INIT.dses_config.mode";
THLA.manager.sim_config.attributes[7].publish      = true;
THLA.manager.sim_config.attributes[7].subscribe    = true;
THLA.manager.sim_config.attributes[7].rti_encoding = THLA_UNICODE_STRING;

//
// Object info
//
THLA.manager.obj_count = 0;

//
// Interaction info
//
THLA.manager.inter_count = 0;
\end{lstlisting}

Lines 2-8 are identical to the input file for {\tt SIM\_simplesine\_pubsub}.
Everything else is new.

Lines 16-18 initialize the federation, supplying values for this federate's
name, the name of the federation to join, the host/port information for the
HLA RTI,\footnote{
  The standard port number is 8989.
  During development the host is often set to localhost, but in general
  the RTI will be running on some agreed location on the network.
}
and the name of the FOM file (located in the same directory as the
\sdefine file).

Lines 20-22 are related to HLA time management: the {\em lookahead time},
and two flags indicating whether the simulation is {\em time regulating}
and {\em time constrained}.
All the examples in this document use a lookahead of just under 1sec
and are both time constrained and time regulating.

Lines 24-28 itemizes a list of {\em known federates}.
This is the \TrickHLA\ mechanism for ensuring that federates wait for
everyone to join before proceeding.
In this example, there is only a single federate ({\tt pubsub\_join}),
but in cases where there are several, the array would be allocated
to hold them all, specifying the name of each and whether they are
required to be present before the others may proceed.

Line 31 is setting the global debug level flags and lead to gradually more or less verbose output.

Lines 38-45 initialize the federation's {\tt SimulationConfiguration} object.
There is one (and only one) of these instances for all the
federates in the federation, but each one (if it uses \TrickHLA)
will nevertheless set these values.
The \TrickHLA\ infrastructure ensures that even though many federates
attempt to publish the object, it only gets created by one of them.
One point worth noting: the value on line 39 for the
{\tt .run\_duration} parameter ensures that the simulation does not run
longer than the specified duration, even if that duration is less than
the value specified with the {\tt STOP} directive (on line 7).

Lines 48-101 are also related to the {\tt SimulationConfiguration} object
as well as multi-phase initialization.
(We do not discuss multi-phase initialization in this example.)

Finally, lines 106 and 111 specify that this simulation has no objects
to publish or subscribe and no interactions to send or receive.
Subsequent examples will elaborate on the \TrickHLA\ publish/subscribe
mechanisms.

\section{Output}

The relevant output for this simulation is the output stream from the
running simulation, which (among other things) verifies that the simulation
did indeed create and join an HLA federation.
An abbreviated version of the output is shown below.

\begin{lstlisting}[caption={{\tt SIM\_simplesine\_hla\_join} output},label={list:SIM-simplesine-hla-join-output}]
...
| |wormhole|1|0.00|2007/07/25,17:59:09| TrickHLAFedAmb::initialize()
| |wormhole|1|0.00|2007/07/25,17:59:09| TrickHLAFederate::initialize()
| |wormhole|1|0.00|2007/07/25,17:59:09| TrickHLAManager::initialize()
...
| |wormhole|1|0.00|2007/07/25,17:59:10|
 TRIVIAL: Trick Federation "simplesine": CREATING FEDERATION EXECUTION
| |wormhole|1|0.00|2007/07/25,17:59:10|
 ADVISORY: Trick Federation "simplesine": SUCCESSFULLY CREATED FEDERATION EXECUTION
| |wormhole|1|0.00|2007/07/25,17:59:10|
 TRIVIAL: Trick Federation "simplesine": JOINING FEDERATION EXECUTION
| |wormhole|1|0.00|2007/07/25,17:59:10|
 ADVISORY: Trick Federation "simplesine": JOINED FEDERATION EXECUTION
Federate Handle = 2
...
| |wormhole|1|0.00|2007/07/25,17:59:10| TrickHLAFederate::wait_for_required_federates_to_join()
WAITING FOR 1 REQUIRED FEDERATES:
    1: Waiting for Federate 'pubsub_join'
...
| |wormhole|1|0.00|2007/07/25,17:59:10| TrickHLAFederate::wait_for_required_federates_to_join()
WAITING FOR 1 REQUIRED FEDERATES:
    1: Found required Federate 'pubsub_join'
...
| |wormhole|1|0.00|2007/07/25,17:59:10| TrickHLAManager::initialization_complete()
        Simulation has started and is now running...
...
Federate "pubsub_join" Time granted to: 1
...
Federate "pubsub_join" Time granted to: 16
...
 TRIVIAL: Trick Federation "simplesine": RESIGNING FROM FEDERATION
...
 ADVISORY: Trick Federation "simplesine": RESIGNED FROM FEDERATION
...
Federation destroyed
...
SIMULATION TERMINATED IN
  PROCESS: 1
  JOB/ROUTINE: 11/sim_services/mains/master.c
DIAGNOSTIC:
Simulation reached input termination time.

LAST JOB CALLED: THLA.THLA.federate.time_advance_request()
              TOTAL OVERRUNS:            0
PERCENTAGE REALTIME OVERRUNS:        0.000%


       SIMULATION START TIME:        0.000
        SIMULATION STOP TIME:       15.000
     SIMULATION ELAPSED TIME:       15.000
         ACTUAL ELAPSED TIME:       15.000
        ACTUAL CPU TIME USED:        0.040
    SIMULATION / ACTUAL TIME:        1.000
       SIMULATION / CPU TIME:      375.056
  ACTUAL INITIALIZATION TIME:        0.000
     INITIALIZATION CPU TIME:        0.719
*** DYNAMIC MEMORY USAGE ***
     CURRENT ALLOCATION SIZE:  1569508
       NUM OF CURRENT ALLOCS:      483
         MAX ALLOCATION SIZE:  1569508
           MAX NUM OF ALLOCS:      483
       TOTAL ALLOCATION SIZE:  1794096
         TOTAL NUM OF ALLOCS:     3301
\end{lstlisting}

Lines 1-14 show the simulation running through its startup process,
initializing the \TrickHLA\ objects which in turn create and join
the HLA federation names {\em simplesine}.

Lines 16-25 show the simulation waiting for all the required federates to join,
which in this case is just this simulation.

Lines 27-29 show the beginning and end of the time grants issues by HLA,
starting at $t=1$ and going thru the end of the simulation duration,
as $t$ extended past the specified limit of 15.

Lines 31-35 show the simulation resigning from the federation and
destroying it (since no federates remain).

The remaining lines are standard Trick output generated at the end of a run.
Note that in spite of the fact that the input file specified
{\tt stop = 32.5} as shown on line 7 of the input file
(Listing~\ref{list:SIM-simplesine-hla-join-input} on page~\pageref{list:SIM-simplesine-hla-join-input}),
the simulation actually terminated earlier due to the
simulation configuration {\tt .run\_duration = 15.0} as specified in
the input file.
