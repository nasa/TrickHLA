%%%%%%%%%%%%%%%%%%%%%%%%%%%%%%%%%%%%%%%%%%%%%%%%%%%%%%%%%%%%%%%%%%%%%%%%%
%
% File: MODELIVV.tex
%
% Purpose: MODEL
%          Inspection, Verification, and Validation
%
%%%%%%%%%%%%%%%%%%%%%%%%%%%%%%%%%%%%%%%%%%%%%%%%%%%%%%%%%%%%%%%%%%%%%%%%%

\newcommand\documentHistory{
{\bf Author} & {\bf Date} & {\bf Description} \\ \hline \hline
YOUR NAME & DATE & Initial Version \\ \hline
}

\newcommand\DocumentChangeHistory{
{\bf Revised by} & {\bf Date} & {\bf Description} \\ \hline \hline

% This documentation file's change history includes:
% "REVISED BY name(s)" & "revision DATE" & "revision DESCRIPTION"
%  ------------------     -------------     --------------------
                       &                 &                    \\ \hline
}

\documentclass[twoside,11pt,titlepage]{report}

%
% Bring in the amsmath package
%
\usepackage{amsmath}

%
% Bring in the common page setup
%
\usepackage{trickhlaenv}

%
% Bring in the common math nomenclature
%
\usepackage{trickhlamath}

%
% Bring in the model-specific commands with requirement labels
%
\usepackage[Reqt]{MODEL}

%
% Bring in the graphics environment
%
\usepackage{graphicx}

%
% Use the float and array packages
%
\usepackage{float, array}

%
% Bring in the hyper ref environment
%
\usepackage[colorlinks]{hyperref}
%  keywords for pdfkeywords are separated by commas
\hypersetup{
   pdftitle={\MODEL\ Inspection, Verification, and Validation},
   pdfauthor={YOUR NAME},
   pdfkeywords={\MODEL, Inspection, Verification, Validation},
   pdfsubject={\MODEL\ Inspection, Verification, and Validation}}

\begin{document}

%%%%%%%%%%%%%%%%%%%%%%%%%%%%%%%%%%%
% Front matter
%%%%%%%%%%%%%%%%%%%%%%%%%%%%%%%%%%%
\pagenumbering{roman}

\docid{DD.mm.05}
\docrev{1.0}
\date{DATE}
\modelname{\MODEL}
\doctype{Inspection, Verification, and Validation}
\author{YOUR NAME}
\managers{
  Dan E. Dexter \\ Project Manager \\
  Leslie J. Quiocho \\ Common Model Lead \\
  Rodney M. Ondler \\ Software, Robotics, and Simulation Division Chief}
\pdfbookmark{Title Page}{titlepage}
\makeTrickhlaenvTitlepage

\pdfbookmark{Abstract}{abstract}
%%%%%%%%%%%%%%%%%%%%%%%%%%%%%%%%%%%%%%%%%%%%%%%%%%%%%%%%%%%%%%%%%%%%%%%%%
%
% Purpose: Abstract for MODEL
%
% Author: YOUR NAME - DATE
% 
% Modified: 
%  
%
%%%%%%%%%%%%%%%%%%%%%%%%%%%%%%%%%%%%%%%%%%%%%%%%%%%%%%%%%%%%%%%%%%%%%%%%%

\begin{abstract}
This is the abstract of the \MODEL.
\end{abstract}


\pdfbookmark{Contents}{contents}
\tableofcontents
\vfill

\pagebreak

%%%%%%%%%%%%%%%%%%%%%%%%%%%%%%%%%%%
% Main Document Body
%%%%%%%%%%%%%%%%%%%%%%%%%%%%%%%%%%%
\pagenumbering{arabic}

%----------------------------------
\chapter{Introduction}\label{sec:intro}
%----------------------------------

%%%%%%%%%%%%%%%%%%%%%%%%%%%%%%%%%%%%%%%%%%%%%%%%%%%%%%%%%%%%%%%%%%%%%%%%%
%
% Purpose: Introduction for MODEL
%
% Author: YOUR NAME - DATE 
% 
% Modified: 
%  
%
%%%%%%%%%%%%%%%%%%%%%%%%%%%%%%%%%%%%%%%%%%%%%%%%%%%%%%%%%%%%%%%%%%%%%%%%%

\MODEL\ introduction. 


\section{Identification of Document}
This document describes the inspection, verification, and validation of
the \MODEL\ developed for use in the Trick Simulation Environment.
This document adheres to the documentation standards defined in
NASA Software Engineering Requirements Standard \cite{NASA:SWE}.

\section{Scope of Document}
This document provides information on
the inspection, verification, and validation of the \MODEL.
This document provides information on the use of the \MODEL.

\section{Purpose and Objectives of Document}
The purpose of this document is to demonstrate that
the \MODEL\ adheres to the requirements levied upon it,
thus enabling the use of the \MODEL\ in a dynamic simulation.

\section{Documentation Status and Schedule}
The information in this document is current with the \TrickHLAid\
implementation of the \MODEL. Updates will be kept current with
module changes.

\begin{tabular}{||l|l|l|} \hline
\documentHistory
\end{tabular}

\begin{tabular}{||l|l|l|} \hline
\DocumentChangeHistory
\end{tabular}

\section{Document Contents}
This document is organized into the following sections:

\begin{description}

\item[Chapter \ref{sec:intro}: Introduction] -
Identifies this document, defines the scope and purpose, present status,
and provides a description of each major section.

\item[Chapter \ref{sec:docs}: Related Documentation] -
Lists the related documentation that is applicable to this project.

\item[Chapter \ref{sec:inspect}: Inspection and Verification] -
Presents the inspection results on the \MODEL.

\item[Chapter \ref{sec:test}: Validation] -
Presents the \MODEL\ test plans and results.

\item[Chapter \ref{sec:traceability}: Requirements Traceability] -
Presents mapping of requirements to inspections and tests.

\item[Bibliography] -
Informational references associated with this document.

\end{description}

\chapter{Related Documentation}\label{sec:docs}

\section{Parent Documents}
The following documents are parent to this document:

\begin{itemize}
\item{\href{file:\TRICKHLAHOME/docs/TrickHLA.pdf}
           {\em Trick High Level Architecture (\TrickHLA)}}
\cite{trickhlaenv:TrickHLA}

\item{\href{file:MODEL.pdf}
           {\em \MODEL}}
\cite{trickhlaenv:MODEL}
\end{itemize}

\section{Applicable Documents}
The following documents are referenced herein and are directly
applicable to this document:

\begin{itemize}
\item{\href{file:MODELReqt.pdf}
           {\em \MODEL\ Product Requirements}}
\cite{trickhlaenv:MODELReqt}

\item{\href{file:MODELSpec.pdf}
           {\em \MODEL\ Product Specification}}
\cite{trickhlaenv:MODELSpec}

\item{\href{file:MODELUser.pdf}
           {\em \MODEL\ User Guide}}
\cite{trickhlaenv:MODELUser}

\item{\em The Trick User's Guide: Trick 2005.0 Release}
\cite{Vetter:TrickUser}

\item{\em Trick Simulation Environment: User Training Materials:
          Trick 2005.0 Release}
\cite{Vetter:TrickUTM}

\item{\em Trick Simulation Environment: Version Description:
          Trick 2005.0 Release}
\cite{Vetter:TrickVD}

\item{\em The Trick Design Document: Trick 2005.0 Release}
\cite{Vetter:TrickDD}

\item{\em NASA Software Engineering Requirements}
\cite{NASA:SWE}

\end{itemize}


\chapter{Inspection and Verification}\label{sec:inspect}

\inspection{Documentation}\label{inspect:documentation}

The \MODEL\ documentation set listed in chapter~\ref{sec:docs}
satisfy requirement~\ref{reqt:documentation}.

\inspection{Trick Coding Standards}\label{inspect:code}
The \MODEL\ source code satisfies the coding requirements
\ref{reqt:h_trick_header},
\ref{reqt:enum_trick_comments},
\ref{reqt:struct_trick_comments},
\ref{reqt:c_trick_header}, and
\ref{reqt:func_trick_comments}.


THIS IS AN EXAMPLE FOR UNIVERSAL TIME.

\inspection{Time Representation}\label{inspect:data_time_representation}
The universal time model contains the data items needed to satisfy
requirement \ref{reqt:data_time_representation}.

\chapter{Validation}\label{sec:test}

\test{Time Initialization}\label{test:func_time_initialization}
\begin{description}
\item[Purpose:] \ \newline
The purpose of this test is to verify that time is properly initialized.
\item[Requirements:] \ \newline
By passing this test, the universal time module partially satisfies
requirement~\ref{reqt:data_time_representation} and completely satisfies
requirement~\ref{reqt:func_time_initialization}.
\item[Procedure:]\ \newline
FILL THIS IN.
\item[Results:]\ \newline
FILL THIS IN.
\end{description}


\chapter{Requirements Traceability}\label{sec:traceability}

\begin{table}[ht]\label{tab:reqt_ivv_xref}
\begin{tabular}{||l|l|l|} \hline
{\bf Requirement} & {\bf Inspection or test} \\ \hline \hline
\ref{reqt:documentation} - Documentation &
  Insp. \ref{inspect:documentation} - Documentation \\
\hline
\ref{reqt:h_trick_header} - Header File Trick Header &
  Insp. \ref{inspect:code} - Trick Coding Standards \\ \hline
\ref{reqt:enum_trick_comments} - Enumeration Comments &
  Insp. \ref{inspect:code} - Trick Coding Standards \\ \hline
\ref{reqt:struct_trick_comments} - Structure Comments &
  Insp. \ref{inspect:code} - Trick Coding Standards \\ \hline
\ref{reqt:c_trick_header} - Source File Trick Headers &
  Insp. \ref{inspect:code} - Trick Coding Standards \\ \hline
\ref{reqt:func_trick_comments} - Function Comments &
  Insp. \ref{inspect:code} - Trick Coding Standards \\ \hline
\ref{reqt:data_time_representation} - Time Representation &
  Insp. \ref{inspect:data_time_representation} - Time Representation, \\
 & Test \ref{test:func_time_initialization} - Time Initialization\\
\hline
\end{tabular}
\caption{Requirements Traceability}
\end{table}


\chapter{Version Description}\label{sec:versions}
This section identifies the versions of the \MODEL\ test items
as described in the current release of the \MODEL\ documentation.

\section{Inventory}
The following items comprise the complete configuration-managed
inventory of the \MODEL:

LIST ALL TEST FILES ASSOCIATED WITH THE MODEL
THAT ARE IN THE RAZOR DATABASE.

\section{Change Status}
This is the basic implementation of the software suite. A change
status is not provided.

\section{Adaptation Data}
This is the first release (basic version) of the \MODEL\ software suite.
No adaptation data is appropriate or is provided, as the user will
define the ab initio data structures at initialization time.

%%%%%%%%%%%%%%%%%%%%%%%%%%%%%%%%%%%%%%%%%%%%%%%%%%%%%%%%%%%%%%%%%%%%%%%%%
% Bibliography
%%%%%%%%%%%%%%%%%%%%%%%%%%%%%%%%%%%%%%%%%%%%%%%%%%%%%%%%%%%%%%%%%%%%%%%%%
\newpage
\pdfbookmark{Bibliography}{bibliography}
\bibliography{trickhlaenv,MODEL}
\bibliographystyle{plain}

\end{document}
