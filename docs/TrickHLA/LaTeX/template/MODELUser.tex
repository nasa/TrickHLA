%%%%%%%%%%%%%%%%%%%%%%%%%%%%%%%%%%%%%%%%%%%%%%%%%%%%%%%%%%%%%%%%%%%%%%%%%
%
% File: MODELUser.tex
%
% Purpose: MODEL User Guide
%
%%%%%%%%%%%%%%%%%%%%%%%%%%%%%%%%%%%%%%%%%%%%%%%%%%%%%%%%%%%%%%%%%%%%%%%%%

\newcommand\documentHistory{
{\bf Author} & {\bf Date} & {\bf Description} \\ \hline \hline
YOUR NAME & DATE & Initial Version \\ \hline
}

\newcommand\DocumentChangeHistory{
{\bf Revised by} & {\bf Date} & {\bf Description} \\ \hline \hline

% This documentation file's change history includes:
% "REVISED BY name(s)" & "revision DATE" & "revision DESCRIPTION"
%  ------------------     -------------     --------------------
                       &                 &                    \\ \hline
}

\documentclass[twoside,11pt,titlepage]{report}

%
% Bring in the AMS math environment
%
\usepackage{amsmath}

%
% Bring in the common page setup
%
\usepackage{trickhlaenv}

%
% Bring in the common math nomenclature
%
\usepackage{trickhlamath}

%
% Bring in the model-specific commands
%
\usepackage{MODEL}

%
% Bring in the graphics environment
%
\usepackage{graphicx}

%
% Bring in the hyper ref environment
%
\usepackage[colorlinks]{hyperref}
%  keywords for pdfkeywords are separated by commas
\hypersetup{
   pdftitle={\MODEL\ User Guide},
   pdfauthor={YOUR NAME},
   pdfkeywords={\MODEL, User Guide},
   pdfsubject={\MODEL\ User Guide}}

\begin{document}

%%%%%%%%%%%%%%%%%%%%%%%%%%%%%%%%%%%
% Front matter
%%%%%%%%%%%%%%%%%%%%%%%%%%%%%%%%%%%
\pagenumbering{roman}

\docid{DD.mm.04}
\docrev{1.0}
\date{DATE}
\modelname{\MODEL}
\doctype{User Guide}
\author{YOUR NAME}
\managers{
  Dan E. Dexter \\ Project Manager \\
  Leslie J. Quiocho \\ Common Model Lead \\
  Rodney M. Ondler \\ Software, Robotics, and Simulation Division Chief}
\pdfbookmark{Title Page}{titlepage}
\makeTrickhlaenvTitlepage

\pdfbookmark{Abstract}{abstract}
%%%%%%%%%%%%%%%%%%%%%%%%%%%%%%%%%%%%%%%%%%%%%%%%%%%%%%%%%%%%%%%%%%%%%%%%%
%
% Purpose: Abstract for MODEL
%
% Author: YOUR NAME - DATE
% 
% Modified: 
%  
%
%%%%%%%%%%%%%%%%%%%%%%%%%%%%%%%%%%%%%%%%%%%%%%%%%%%%%%%%%%%%%%%%%%%%%%%%%

\begin{abstract}
This is the abstract of the \MODEL.
\end{abstract}


\pdfbookmark{Contents}{contents}
\tableofcontents
\vfill

\pagebreak

%%%%%%%%%%%%%%%%%%%%%%%%%%%%%%%%%%%
% Main Document Body
%%%%%%%%%%%%%%%%%%%%%%%%%%%%%%%%%%%
\pagenumbering{arabic}

%----------------------------------
\chapter{Introduction}\label{sec:intro}
%----------------------------------

%%%%%%%%%%%%%%%%%%%%%%%%%%%%%%%%%%%%%%%%%%%%%%%%%%%%%%%%%%%%%%%%%%%%%%%%%
%
% Purpose: Introduction for MODEL
%
% Author: YOUR NAME - DATE 
% 
% Modified: 
%  
%
%%%%%%%%%%%%%%%%%%%%%%%%%%%%%%%%%%%%%%%%%%%%%%%%%%%%%%%%%%%%%%%%%%%%%%%%%

\MODEL\ introduction. 


\section{Identification of Document}
This document describes the use of the
\MODEL\ developed for use in the Trick Simulation Environment.
This document adheres to the documentation standards defined in
NASA Software Engineering Requirements Standard \cite{NASA:SWE}.

\section{Scope of Document}
This document provides information on the use of the \MODEL.

\section{Purpose and Objectives of Document}
The purpose of this document is to describe how to incorporate the
\MODEL\ into a dynamic Trick simulation and used by other simulation models.

\section{Documentation Status and Schedule}
The information in this document is current with the \TrickHLAid\
implementation of the \MODEL. Updates will be kept current with
module changes.

\begin{tabular}{||l|l|l|} \hline
\documentHistory
\end{tabular}

\begin{tabular}{||l|l|l|} \hline
\DocumentChangeHistory
\end{tabular}

\section{Document Contents}
This document is organized into the following sections:

\begin{description}

\item[Chapter \ref{sec:intro}: Introduction] -
Identifies this document, defines the scope and purpose, present status,
and provides a description of each major section.

\item[Chapter \ref{sec:docs}: Related Documentation] -
Lists the related documentation that is applicable to this project.

\item[Chapter \ref{sec:guide}: User Guide] -
Describes how to use the \MODEL.

\item[Bibliography] -
Informational references associated with this document.

\end{description}

\chapter{Related Documentation}\label{sec:docs}

\section{Parent Documents}
The following documents are parent to this document:

\begin{itemize}
\item{\href{file:\TRICKHLAHOME/docs/TrickHLA.pdf}
           {\em Trick High Level Architecture (\TrickHLA)}}
\cite{trickhlaenv:TrickHLA}

\item{\href{file:MODEL.pdf}
           {\em \MODEL}}
\cite{trickhlaenv:MODEL}
\end{itemize}

\section{Applicable Documents}
The following documents are referenced herein and are directly
applicable to this document:

\begin{itemize}
\item{\href{file:MODELReqt.pdf}
           {\em \MODEL\ Product Requirements}}
\cite{trickhlaenv:MODELReqt}

\item{\href{file:MODELSpec.pdf}
           {\em \MODEL\ Product Specification}}
\cite{trickhlaenv:MODELSpec}

\item{\href{file:MODELIVV.pdf}
           {\em \MODEL\ Inspection, Verification, and Validation}}
\cite{trickhlaenv:MODELIVV}

\item{\em The Trick User's Guide: Trick 2005.0 Release}
\cite{Vetter:TrickUser}

\item{\em Trick Simulation Environment: User Training Materials:
          Trick 2005.0 Release}
\cite{Vetter:TrickUTM}

\item{\em Trick Simulation Environment: Version Description:
          Trick 2005.0 Release}
\cite{Vetter:TrickVD}

\item{\em The Trick Design Document: Trick 2005.0 Release}
\cite{Vetter:TrickDD}

\item{\em NASA Software Engineering Requirements}
\cite{NASA:SWE}

\end{itemize}

\chapter{User Guide}\label{sec:guide}
This section describes how to use the
\MODEL\ in a simulation and in other models.

\section{Simulation}

\subsection{S\_define File}

USE SNIPPETS OF AN S\_DEFINE TO ILLUSTRATE HOW THE MODEL
CAN BE INCORPORATED INTO A SIMULATION.

\subsection{Input Files}

DESCRIBE WHICH MODEL DATA ARE TYPICALLY CHANGED IN AN INPUT FILE.
USE EXAMPLES IF APPLICABLE.

\subsection{Log Files}

DESCRIBE WHICH MODEL DATA ARE TYPICALLY LOGGED.
USE EXAMPLES IF APPLICABLE.

%%%%%%%%%%%%%%%%%%%%%%%%%%%%%%%%%%%%%%%%%%%%%%%%%%%%%%%%%%%%%%%%%%%%%%%%%
% Bibliography
%%%%%%%%%%%%%%%%%%%%%%%%%%%%%%%%%%%%%%%%%%%%%%%%%%%%%%%%%%%%%%%%%%%%%%%%%
\newpage
\pdfbookmark{Bibliography}{bibliography}
\bibliography{trickhlaenv,MODEL}
\bibliographystyle{plain}

%%%%%%%%%%%%%%%%%%%%%%%%%%%%%%%%%%%%%%%%%%%%%%%%%%%%%%%%%%%%%%%%%%%%%%%%%
% Appendices
%%%%%%%%%%%%%%%%%%%%%%%%%%%%%%%%%%%%%%%%%%%%%%%%%%%%%%%%%%%%%%%%%%%%%%%%%
\newpage
\pdfbookmark{Appendix}{appendix}
\appendix

\chapter{Model Data}\label{sec:model_data}

%\input{xml/MODEL.h.tex}


\chapter{Model Functions}\label{sec:model_functions}

%\input{xml/MODEL_c_file.tex}

\end{document}
