
   \section{Simulation Multiphase Initialization}

   \begin{frame}
      \frametitle{Simulation Multiphase Initialization}
      \begin{center}
      \Huge{Simulation Multiphase Initialization}
      \end{center}
   \end{frame}
   
   \begin{frame}[fragile]
      \frametitle{Simulation Multiphase Initialization}
      \begin{itemize}
         \item TrickHLA supports multiphase simulation initialization where data
         is exchanged between federates before the simulation starts running.
         \item An attribute is used for multiphase simulation initialization by
         specifying \texttt{THLA\_INITIALIZE} for the attribute’s config field
         (as in the previous example shown in the \texttt{RUN\_a\_side/input.py}
         file) :
\begin{Verbatim}[frame=single, fontsize=\scriptsize]
THLA.manager.objects[1].attributes[7].config = trick.THLA_INITIALIZE
                                               + trick.THLA_CYCLIC
\end{Verbatim}
      \end{itemize}
   \end{frame}
   
   \begin{frame}[fragile]
      \frametitle{Simulation Multiphase Initialization}
      \framesubtitle{\texttt{S\_define}: Public data and constructor}
      \begin{itemize}
         \item This is similar to our generic \texttt{THLA\_INIT} simulation object, but
         using a multiphase approach:
\begin{Verbatim}[frame=single, fontsize=\tiny]
. . .
// Include the simple simulation configuration object definition.
##include "simconfig/include/SimpleSimConfig.hh”
. . .

//=============================================================================
// SIM_OBJECT: THLA_INIT  (TrickHLA multi-phase initialization sim-object)
//=============================================================================
class THLAInitSimObj : public Trick::SimObject {

 public:

   TrickHLA::SimTimeline      sim_timeline;
   TrickHLA::ScenarioTimeline scenario_timeline;

   THLAInitSimObj( TrickHLA::Manager  & thla_mngr,
                   TrickHLA::Federate & thla_fed ) 
      : scenario_timeline( sim_timeline, 0.0, 0.0 ),
        thla_manager( thla_mngr ),
        thla_federate( thla_fed )
   {
\end{Verbatim}
      \end{itemize}
   \end{frame}
   
   \begin{frame}[fragile]
      \frametitle{Simulation Multiphase Initialization}
      \framesubtitle{\texttt{S\_define}: Public scheduled jobs}
      \begin{itemize}
         \item Declare the initialization jobs.
\begin{Verbatim}[frame=single, fontsize=\tiny]
      //------------------------------------------------------------------------
      // NOTE: Initialization phase numbers must be greater than P60 
      // (i.e. P_HLA_INIT) so that the initialization jobs run after the
      // P60 THLA.manager->initialize() job.
      //------------------------------------------------------------------------
      // Data will only be sent if this federate owns it.
      P100 ("initialization") thla_manager.send_init_data( "A-side-Federate.Test" );
      
      // Data will only be received if it is remotely owned by another federate.
      P100 ("initialization") thla_manager.receive_init_data( "A-side-Federate.Test" );
      
      // Do some processing here if needed...
      
      // Wait for all federates to reach this sync-point.
      P100 ("initialization") thla_manager.wait_for_init_sync_point( "Phase1" );
      
      // Data will only be sent if this federate owns it.
      P200 ("initialization") thla_manager.send_init_data( "P-side-Federate.Test" );
      
      // Data will only be received if it is remotely owned by another federate.
      P200 ("initialization") thla_manager.receive_init_data( "P-side-Federate.Test" );
      
      // Do some processing here if needed...
      
      // Wait for all federates to reach this sync-point.
      P200 ("initialization") thla_manager.wait_for_init_sync_point( "Phase2" );
   }
\end{Verbatim}
         \end{itemize}
   \end{frame}
   
   \begin{frame}[fragile]
      \frametitle{Simulation Multiphase Initialization}
      \framesubtitle{\texttt{S\_define}: Private data and instantiation}
      \begin{itemize}
         \item Protect from copies and do not permit a default constructor.
\begin{Verbatim}[frame=single, fontsize=\tiny]
      
 private:
   TrickHLA::Manager  & thla_manager;
   TrickHLA::Federate & thla_federate;
 
   // Do not allow the implicit copy constructor or assignment operator.
   THLAInitSimObj(const THLAInitSimObj & rhs);
   THLAInitSimObj & operator=(const THLAInitSimObj & rhs);
   
   // Do not allow the default constructor.
   THLAInitSimObj();
};
. . .
// Intantiate the initialization simulation object.
THLAInitSimObject THLA_INIT( THLA.manager, THLA.federate );
. . .
\end{Verbatim}
      \end {itemize}
   \end{frame}
